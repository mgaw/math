\documentclass{scrartcl}

\usepackage{amsmath}
\usepackage{amssymb}
\usepackage[T1]{fontenc}
\usepackage[utf8]{inputenc}
\usepackage[ngerman]{babel}
\usepackage[left=1cm, right=1cm, top=0.5cm, bottom=0.5cm, a4paper]{geometry}
\usepackage{graphicx}
\usepackage{rotating}

\parindent 0pt
\parskip 6pt

\newcommand{\gdw}{\quad :\Leftrightarrow\quad &}
\newcommand{\is}{\quad :=\,\quad &}
\newcommand{\gleich}{\quad =\;\quad &}
\newcommand{\dann}{\quad \Rightarrow\quad &}
\newcommand{\aeq}{\quad \Leftrightarrow\quad &} % äquivalent
\newcommand{\zb}{\quad \;\text{zB}\quad &}
\newcommand{\teilm}{\subseteq}
\newcommand{\sse}{\subseteq}
\newcommand{\NN}{\mathbb{N}}
\newcommand{\RR}{\mathbb{R}}
\newcommand{\s}[1]{\begin{split}#1\end{split}}
\renewcommand{\r}[1]{\begin{split}#1\end{split} \\}

%TODO: Leichter in Math-Mode kommen
%TODO: "ker(" übersetzen in "\text{ker}(", "sgn(", " und ", "ZZ" in "\mathbb{Z}"
%TODO: weniger "\" tippen müssen. Vll, in dem "sigma", "forall" etc. automatisch eins vorgesetzt
%wird?
%TODO: Sachen außerhalb des Math-Mode in Markdown.
%TODO: /^[A-Z]\S*\.\s/ Solche hervorheben.
%TODO: griechische Buchstaben dann direkt als Unicode eingeben, ggf. mit :ab \sigma σ 

\begin{document}

\section{20.11.2012}

Prop. Seien $a, b \in \mathbb{Z}, d = ggT(a, b)$. Dann gibt es $x, y \in \mathbb{Z}$, so dass $d=xa+yb$.

Bmk. Der ggt($a, b$) ist die kleinste natürliche Zahl, die als lineare Kombination von a und b
darstellbar ist.

Lemma. Sei $f:G \to G'$ Gruppenmorphismus. Dann ist $N:=ker(f) \lhd G$.

Def. $sgn(\sigma) = (-1)^{#\{(i,j): i<j und \sigma(i) > \sigma(j)\}}$

Lemma. $\forall \sigma \in S_n : sgn(\sigma) = (Produkt) \frac{\sigma(j) - \sigma(i)}{j-i}$.

Def. $sigma in S_n$ heißt gerade, gdw. $sgn(sigma) = 1$. Ungerade, gdw. $sgn(sigma) = -1$.

Def. $Ker(sgn) = A_n = \{ sigma in S_n : sng(sigma) = 1 \}$ heißt die \emph{alternierende Gruppe}.

Bmk. $A_n \lhd S_n$. $|A_n| = \frac{n!}{2}$.

Isomorphiesatz. Sei $phi : G to G'$ Gruppenmorphismus. Dann ist $Ker(phi) \lhd G$ und es gibt einen
kanonischen Isomorphismus von Gruppen $overline φ : G/Ker(φ) gleichschlange to φ(G)$. ($φ(G) <= G'$
Untergruppe.)

Bmk. Falls $φ$ surjektiv ist, d.h. $φ(G) = G'$, dann bildet $overline φ$ nach $G'$ ab.

\end{document}
